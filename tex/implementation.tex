\section{Implementation}
Activities are opened and closed by starting and finishing intents. In certain activities where the user should be able to return to the main menu and it is not the previous activity on the stack, a new intent to the main menu is created with the added flag \verb|FLAG_ACTIVITY_CLEAR_TOP|. This brings the very first activity (MainActivity) to the top and closes the others. Another used flag in the application is \verb|FLAG_ACTIVITY_NO_ANIMATION|, which removes the animation during the transition between PhotoActivity and GameActivity. To remove the animation when an activity is closed, the ‘finish’ method is overriden with the added line “overridePendingTransition(0, 0)”. \newline

The only activity that is started with startActivityForResult is the camera application. When a photo is taken, onActivityResult is called with a request code, a result code and data. If the request code and result code are correct, a new intent to GameActivity is created with the data added. In GameActivity, the data can now be used to get the photo that the user has taken and apply it to the bonus card. The application checks whether the user has taken a photo or wants to use the default card with getStringExtra. \newline

The key word is based on the type of gesture the user is done. A bottom swipe has the key “keep”, an up swipe has the key “skip” and double tap has either the key “bonus” or “rush” if a bonus or rush card.  The VerifyFragment class creates an interface to be able to interact with the GameActivity class, in order to the reference photo and the current photo. These images are then sent through the AsyncTask where the photos id’s check to be same or not.  If the images are the same and the key is keep, then the integer addPoints is set to 1 and the boolean updateAllCards is set to true. If the images are not the same and the key is keep addPoints is set to -1 and updateAllCards is set to false. After the imageMatch is completed addPoints and updateAllCards will be sent back to the GameActivity and the score and cards are updated accordingly. If the key is “skip” addPoints will be 0 and updateAllCards is set to false. If the key is “bonus” addPoints is set to 5 and updateAllCards to true, and if the key is “rush” the rush game mode starts. Within this game mode addPoints will set to 1 and updateAllCards to false no matter if it is a up or down swipe. \newline

In the FinishDialogFragment we use a AlertDialog.Builder to built a dialog. The dialog have different text and buttons depending on the score that user has got during the game. If the score is a high score the text is set to “Congratulations! You made your highscore“ or “Congratulations! You beat your highscore” if it is the highest high score. If the score is a high score an editText box is added to the dialog where the user can enter his or hers name and a submit button that will store the score and the name in a Shared Preferences. If the score is not a high score the message text is set to “You scored: + score”.
The FinishDialogFragment also uses an interface to interact with the game activity if the user presses the start button. \newline

The image class extends the ImageView class, where we have overwriten the constructor to accept a custom ID, an image resource or bitmap, a bonus boolean and a rush hour boolean. The remaining functions within the image class are getter methods for usage with the GameActivity, GestureListener and VerifyFragment. getDrawImage() returns the image resource stored within the object and getBitmap() returns the bitmap. In the GameActivity getBitmap is used as a check if an image has a bitmap and uses it for ressources otherwise it will uses getDrawImage(). isBonus() and isRush() is used to access the corresponding booleans for usage within the GestureListener to check is the card is a rush or bonus card. getID() is used with the VerifyFragment to get two Images in the Image array’s ID for the idMatch check. \newline

The GestureListener class extends GestureDetector.SimpleOnTouchListener that implements two types of gestures fling and double tap. onFling() method checks what type of swipe the user has used. If it is a topSwipe then the VerifyFragment starts() method is called with the key “skip”. Else if it is a bottomSwipe then the VerifyFragment starts method is also called with the key “keep”. The topSwipe and bottomSwipe check is the coordinates from the two events difference is greater than 120 and the velocity is greater than 200. The only difference between the two methods is the subtraction of between the two events. onDoubleTap checks GameActivity’s current image bonus boolean and rush boolean. If the bonus boolean is true the VerifyFragments start() method is called with the key “bonus” else if the rush boolean is true the start() method is called with the key “rush”. \newline

In HighscoreActivity a top 5 list is shown with the five highest points since last reset. The highscore is kept in a SharedPreferences-class to make sure, they still appear even when destroying and creating the application. The SharedPreferences keep the information as key-value pairs. These pairs have a key to both the name and the points that indicates the place in the highscore and the type. For example will key “name1” be the name of the first place in the highscore, and key “point5” will be the points of the fifth place in the highscore. These keys match the id of each TextView shown in the \verb|activity_highscore.xml| file. Every time the HighScoreActivity is called, the activity updates the names and points, so the values of the specific keys matches the ranking. For this an “insertion sort”-algorithm is used to sort the points, and both names and points are saved in two separate arrays. These arrays are then shown as TextViews in a GridLayout in the \verb|activity_highscore.xml| file.
The reset button clears the Editor in the specific SharedPreference-class and calls onResume to update the view.  \newline

The InstructionsActivity displays a background image, a menu title, an image with the instructions, a text to the corresponding image, a page (image) number, and a back button. It has an array of eight different images and an array of the text corresponding to each of the images, displaying only one image and text section at a time. The initial image is the first image in the array. When swiping to the left, the next image in the array is displayed if the last image is not reached, and swiping to the right, displays the previous image if the first image is not being displayed. The gesture detection is set up via gestureHandling. The swipe is detected via the CustomGestureDetector class, which extends GestureDetector.SimpleOnGestureListener. When the detected horizontal swipe distance and speed are greater than certain values, the corresponding methods increasing or decreasing the image number in the array and the update method is called, setting the correct image, the correct corresponding text, and page number. \newline

The timer is implemented as a CountDownTimer set to 60 seconds. The text view showing the time is updated in the timer’s ‘OnTick’-method.  When the timer reaches 0, the FinishDialogFragment appears. When the rush hour card is double tapped the boolean rushTime is set to true. If the rushTime is true, the text view jumps to 5 seconds while the timer is still running as before. When the timer hits 0, the text view shows the actual time with 7 seconds in addition. This makes it look like two different timers where the first timer continues after the second time has ended. After the actual timer terminates, a new timer is created to show the remaining seconds. It will now look like the timer ends before it reaches 0. We have done it in this way, because it is not possible to have two timers running at the same time. 


